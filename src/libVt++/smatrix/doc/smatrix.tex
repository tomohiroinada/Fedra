\documentclass{article}
\usepackage{verbatim,color,array}
\usepackage{graphicx}

\newcommand{\tcb}[1]{\textcolor{blue}{#1}}
\newcommand{\tcr}[1]{\textcolor{red}{#1}}
\newenvironment{code}[1]{\color{#1}\verbatim}{\endverbatim\color{black}}
\newcommand{\BEE}{\texttt{BEE}}
\newcommand{\SMatrix}{\texttt{SMatrix}}
\newcommand{\SVertex}{\texttt{SVertex}}

\title{SMatrix - A high performance library for Vector/Matrix calculation and
  Vertexing\\[1.5ex]
  \normalsize{Version 0-20}\\[1.5ex]
  \normalsize{HERA-B Note 01-134, Software 01-017 (UPDATED)}}

\author{Thorsten Glebe\\
        Max-Planck-Institut f\"ur Kernphysik, Heidelberg\\
        T.Glebe@mpi-hd.mpg.de}
%\author{T. Glebe}

\begin{document}
\maketitle
\abstract{SMatrix is a C++ based library for high performance vector and matrix
  computations. In addition the \SVertex\  class is provided, which is a
  reimplementation of the \texttt{Vt++} library using the high performance
  vector and matrix classes. With SVertex objects, Kalman filter based vertex
  fits can be performed 20 times faster than with \texttt{Vt++} \cite{Vt++} and
  5-10 times faster than with the \texttt{Grover} \cite{Grover} vertexing
  package. The \SMatrix\ library is part of the \BEE\ analysis framework
  \cite{bee}.}
\tableofcontents
%
% Section
%
\section{Introduction}
\label{intro}
One of the main tasks of data analysis code in high energy physics experiments
is to reconstruct decay chains and to compute properties of particles from a
huge amount of measurements. In many cases the computations involve vectors
(for e.g. track momentum or vertex position) and matrices (e.g. covariance
matrices in a Kalman filter algorithm). Vector and matrix algebra can be
perfectly implemented with an object oriented language like C++, which allows
with its abstraction capabilities to arrive at simple and readable code. In
recent years there has been a proliferation of libraries which provides
abstractions for a wide range of numerical problems.

However, the code generated by such libraries tends to be naive. For example,
array objects implemented using operator overloading in C++ were originally
3-20 times slower than the corresponding low-level implementation. This was not
because of poor design on the part of library developers, but rather because
the language forced a style of implementation which was grossly inefficient.
These performance problems are commonly called the \emph{abstraction penalty}
\cite{templ1}.

A promising approach to overcome the described problem is to construct
libraries which provides both the abstractions, and control how they are
optimized. This concept has been called \emph{active libraries} \cite{alib}.
\SMatrix\ is such a library, it handles high-level optimizations
themselve and leaves only low-level optimization (register allocation,
instruction scheduling, software pipelining) to the optimizer.

The introduction of templates to C++ added a facility whereby the compiler can
act as an interpreter. This makes it possible to write programs in a subset of
C++ which are interpreted at compile time. This technique is called
\emph{template metaprogramming} \cite{tmplmeta} and is used to
achieve the above mentioned high-level optimizations. C++ templates are a
programming language by its own and can be used to implement vector and matrix
expressions such that these expressions can be transformed at compile time to
code which is equivalent to hand optimized code in a low-level language like FORTRAN
or C \cite{exprtmpl, exprtmpl2}.

The \SMatrix\ library has been written to exploit the template techniques for a
high performance implementation of a Kalman filter algorithm for vertex
fitting. Investigations of typical analysis jobs have shown that this is the
place where 70-90\% of the computing time is spent.

The \SMatrix\ library is fully templated, that means it is sufficient to
include the appropriate header files. No library to link with is needed.
%
% Section
%
\section{Vectors}
The \texttt{SVector} class represents $n$-dimensional vectors for objects of
arbitrary type. The type (usually \texttt{float}, \texttt{double} or
\texttt{int}) as well as the vector dimension must be given as a template
argument:
\begin{code}{blue}
  #include <iostream>
  #include "SVector.hh"
  
  int main() {
    
    SVector<float,3> x(1,2,3); // 3-dimensional
    SVector<float,3> y(4,5,6); // 3-dimensional

    cout << "x + y: " << x + y << endl;

    return 0;
  }
\end{code}
The result is, as expected:
\begin{code}{blue}
x + y: 5, 7, 9 
\end{code}
The $[]$ and $()$ operators can be used to access the individual elements of a
vector:
\begin{code}{blue}
  SVector<float,3> a;
  a[0] = 1.; a[1] = 2.; a(2) = 3.;
  cout << "Elements: " << a[0] << ", " << a(1) << ", " << a[2] 
       << endl;
\end{code}
Defined operators are \texttt{+}, \texttt{-} (both binary and unary),
\texttt{*}, \texttt{/}, \texttt{+=}, \texttt{-=}, \texttt{*=}, \texttt{/=},
\texttt{fabs} (absolute values), \texttt{sqr} (square) and \texttt{sqrt}
(square root):
\begin{code}{blue}
  SVector<float,3>    x(1,2,3);
  SVector<float,3>    y(4,5,6);
  SVector<float,3>    z(4,16,64);

  // element wise multiplication
  cout << "x * y: " << x * -y << endl;

  x += z - y;
  cout << "x += z - y: " << x << endl;

  // element wise square root
  cout << "sqrt(z): " << sqrt(z) << endl;

  // element wise multiplication with constant
  cout << "2 * y: " << 2 * y << endl;

  // a more complex expression
  cout << "fabs(-z + 3*x): " << fabs(-z + 3*x) << endl;
\end{code}
The output is:
\begin{code}{blue}
x * y: -4, -10, -18
x += z - y: 1, 13, 61
sqrt(z): 2, 4, 8
2 * y: 8, 10, 12
fabs(-z + 3*x): 1, 23, 119
\end{code}
Special operators are: \texttt{dot} (scalar product), \texttt{mag2} (squared
norm), \texttt{mag} (norm), \texttt{cross} (cross product) and \texttt{unit}
(return vector of length 1):
\begin{code}{blue}
  SVector<float,3>    x(1,2,3);
  SVector<float,3>    y(4,5,6);

  // scalar product
  cout << "dot(x,y): " << dot(x,y) << endl;

  // norm of x
  cout << "mag(x): " << mag(x) << endl;

  // cross product of x and y
  cout << "cross(x,y): " << cross(x,y) << endl;

  // copy of x, rescaled to lenght = 1
  cout << "unit(x): " << unit(x) << endl;
\end{code}
The output is:
\begin{code}{blue}
dot(x,y): 32
mag(x): 3.74166
cross(x,y): -3, 6, -3
unit(x): 0.267261, 0.534522, 0.801784
\end{code}
\textbf{Note:} the cross product works only for vectors with dimension $n=3$.

A vector $\vec{a}$ can be placed in another vector $\vec{b}$ via the
\texttt{place\_at} member function. For the vector dimensions the condition
$\texttt{dim(}\vec{b}\texttt{)} \ge \texttt{dim(}\vec{a}\texttt{)}$ must hold.
\begin{code}{blue}
  SVector<float,4> a;
  SVector<float,2> b(1,2);

  a.place_at(b,2);
  cout << "a: " << a << endl;
\end{code}
The output is:
\begin{code}{blue}
a: 0, 0, 1, 2
\end{code}
A big advantage of coding the vector dimension as a template parameter is that
the correctness of dimensions in expressions can be checked already at compile
time. In case of vector/matrix classes with dynamic dimensions the correctness
of expressions can only be detected during run time\footnote{In the best case
  a segmentation violation occurs, in the worst case the result is just wrong.}.
%
% Section
%
\section{Matrix}
The \SMatrix\ class represents matrices with $n\times m$ dimensions of
arbitrary type. Again, the type as well as the matrix dimensions must be
specified as template arguments:
\begin{code}{blue}
  #include <iostream>
  #include <SMatrix>

  int main() {

    SMatrix<float,2,3> A = 15.; // 2x3 matrix
  
    cout << "A: " << endl << A << endl;

    return 0;
  }
\end{code}
The output is:
\begin{code}{blue}
A:
[           15          15          15
            15          15          15 ]
\end{code}
In case of an $n\times n$ matrix it is sufficient to specify the dimension only
once:
\begin{code}{blue}
    SMatrix<float,3> A = 15.; // 3x3 matrix
  
    cout << "A: " << endl << A << endl;
\end{code}
The output is:
\begin{code}{blue}
A:
[           15          15          15
            15          15          15
            15          15          15 ]
\end{code}
Matrix elements can be accessed via the \texttt{()} operator:
\begin{code}{blue}
    SMatrix<float,2,3> A = 15.; // 2x3 matrix
    A(0,0) = A(1,1) = A(0,2) = 5.;
\end{code}
The output is:
\begin{code}{blue}
A:
[            5          15           5
            15           5          15 ]
\end{code}
\textbf{Note:} there is no check whether the indices point to a valid matrix
element. The \SMatrix\ class is designed for high performance and not
for fool-proofness.

A row or a column of a \SMatrix\ object can be returned as a
\texttt{SVector} object:
\begin{code}{blue}
  SMatrix<double,3> A;
  A(0,0) = A(1,0) = 1;
  A(0,1) = 3;
  A(1,1) = A(2,2) = 2;
  cout << "A: " << endl << A << endl;

  // 1st row
  SVector<double,3> x = A.row(0);
  cout << "x: " << x << endl;

  // 2nd column
  SVector<double,3> y = A.col(1);
  cout << "y: " << y << endl;
\end{code}
The output is:
\begin{code}{blue}
A:
[            1           3           0
             1           2           0
             0           0           2 ]
x: 1, 3, 0
y: 3, 2, 0
\end{code}
Vector or vector expressions can be placed inside a matrix. This is useful if
you want to compose matrices from different (sub-) expressions. The
\texttt{place\_in\_row} member function places vectors in matrix rows,
\texttt{place\_in\_col} places vectors in matrix columns. A smaller matrix can be
placed inside a bigger matrix via the \texttt{place\_at} member function:
\begin{code}{blue}
  SVector<float,3> x(4,5,6);
  SVector<float,2> y(2,3);
  cout << "x: " << x << endl;
  cout << "y: " << y << endl;

  SMatrix<float,4,3> A;
  SMatrix<float,2,2> B = 1;

  A.place_in_row(y, 1, 1);
  A.place_in_col(x + 2, 1, 0);
  A.place_at(B * -4, 2, 1);
  cout << "A: " << endl << A << endl;
\end{code}
The output is:
\begin{code}{blue}
x: 4, 5, 6
y: 2, 3
A:
[            0           0           0
             6           2           3
             7          -4          -4
             8          -4          -4 ]
\end{code}
%
% Subsection
%
\subsection{Matrix - Matrix expressions}
Defined operators are \texttt{+}, \texttt{-} (both binary and unary),
\texttt{times}, \texttt{/}, \texttt{+=}, \texttt{-=}, \texttt{*=}, \texttt{/=},
\texttt{fabs} (absolute values), \texttt{sqr} (square) and \texttt{sqrt}
(square root). All these operators work element wise (are applied to each
element of the matrix):
\begin{code}{blue}
  SMatrix<float,2,3>  x = 4.;
  SMatrix<float,2,3>  y = 5.;
  SMatrix<float,2,3>  z = 64.;

  // element wise multiplication
  cout << "x * y: " << endl << times(x, -y) << endl;

  x += z - y;
  cout << "x += z - y: " << endl << x << endl;

  // element wise square root
  cout << "sqrt(z): " << endl << sqrt(z) << endl;

  // element wise multiplication with constant
  cout << "2 * y: " << endl << 2 * y << endl;

  // a more complex expression
  cout << "fabs(-z + 3*x): " << endl << fabs(-z + 3*x) << endl;
\end{code}
The output is:
\begin{code}{blue}
x * y:
[          -20         -20         -20
           -20         -20         -20 ]
x += z - y:
[           63          63          63
            63          63          63 ]
sqrt(z):
[            8           8           8
             8           8           8 ]
2 * y:
[           10          10          10
            10          10          10 ]
fabs(-z + 3*x):
[          125         125         125
           125         125         125 ]
\end{code}
Matrix multiplication is done via the \texttt{*} operator:
\begin{code}{blue}
  SMatrix<float,4,2> A;
  A(0,0) = A(0,1) = A(1,1) = A(2,0) = A(3,1) = 4.;
  cout << "A: " << endl << A << endl;
  
  SMatrix<float,2,3> S;
  S(0,0) = S(0,1) = S(1,1) = S(0,2) = 1.;
  cout << " S: " << endl << S << endl;
  
  // matrix multiplication
  SMatrix<float,4,3> C = A * S;
  cout << " C: " << endl << C << endl;
\end{code}
The output is:
\begin{code}{blue}
A:
[            4           4
             0           4
             4           0
             0           4 ]
 S:
[            1           1           1
             0           1           0 ]
 C:
[            4           8           4
             0           4           0
             4           4           4
             0           4           0 ]
\end{code}
%
% Subsection
%
\subsection{Vector - Matrix expressions}
Expressions including \SMatrix\ and \texttt{SVector} objects are
allowed, too. The matrix - vector multiplication is implemented via the
\texttt{*} operator:
\begin{code}{blue}
  SMatrix<float,4,3> A;
  A(0,0) = A(0,1) = A(1,1) = A(2,2) = 4.;
  A(2,3) = 1.;
  cout << "A: " << endl << A << endl;
  SVector<float,4> x(1,2,3,4);
  cout << " x: " << x << endl;
  SVector<float,3> a(1,2,3);
  cout << " a: " << a << endl;

  // Matrix-Vector expression
  SVector<float,4> y = x + A * a;
  cout << " y: " << y << endl;
\end{code}
The output is:
\begin{code}{blue}
A:
[            4           4           0
             0           4           0
             0           0           4
             1           0           0 ]
 x: 1, 2, 3, 4
 a: 1, 2, 3
 y: 13, 10, 15, 5
\end{code}
The operators are able to work with expressions, too (take above definitions of
matrix $A$ and vector $x$):
\begin{code}{blue}
  // we add 1 to each component of x and A
  SVector<float,3> b = (x+1) * (A+1);
  cout << " b: " << b << endl;
\end{code}
The output is:
\begin{code}{blue}
 b: 27, 34, 30
\end{code}
A special operator is the \texttt{product} operator, which calculates
expressions of type $\alpha = \vec{a}^{T}\cdot A\cdot\vec{a}$:
\begin{code}{blue}
  SMatrix<double,3> A;
  A(0,0) = A(0,1) = A(1,0) = 1;
  A(1,1) = A(2,2) = 2;
  cout << " A: " << endl << A << endl;

  SVector<double,3> x(1,2,3);
  cout << "x: " << x << endl;

  // we add 1 to each component of x and A
  cout << " (x+1)^T * (A+1) * (x+1): " << product(x+1,A+1) << endl;
\end{code}
The output is:
\begin{code}{blue}
 A:
[            1           1           0
             1           2           0
             0           0           2 ]
x: 1, 2, 3
 (x+1)^T * (A+1) * (x+1): 147
\end{code}
%
% Subsection
%
\subsection{Matrix member functions}
The following \SMatrix\ member functions for special calculations are
provided:
\begin{description}
  \item[\texttt{det}] calculate determinant of a square matrix
  \item[\texttt{sdet}] calculate determinant of a symmetric, positive definite
    matrix
  \item[\texttt{invert}] invert a square matrix
  \item[\texttt{sinvert}] invert a symmetric, positive definite matrix
\end{description}
\textbf{Note:} all these member functions will modify the contents of the
matrix. In case of \texttt{invert()} and \texttt{sinvert()} the matrix is
transformed into its inverse.
Example:
\begin{code}{blue}
  SMatrix<double,3> A;
  A(0,0) = A(0,1) = A(1,0) = 1;
  A(1,1) = A(2,2) = 2;
  cout << "A: " << endl << A << endl;

  double det = 0.;
  // determinant of symmetric, pos. def. matrix
  A.sdet(det);
  cout << "Determinant: " << det << endl;

  // WARNING: A has changed!!
  cout << "A again: " << endl << A << endl;
\end{code}
The output is:
\begin{code}{blue}
A:
[            1           1           0
             1           2           0
             0           0           2 ]
Determinant: 2
A again:
[            1           1           0
             1           1           0
             0           0         0.5 ]
\end{code}
The \texttt{det()} and \texttt{sdet()} functions are derived from the CERNLIB
functions \texttt{dfact} and \texttt{dsfact}�\cite{cernlib}.

Example for matrix inversion:
\begin{code}{blue}
  SMatrix<double,3> A;
  A(0,0) = A(0,1) = A(1,0) = 1;
  A(1,1) = A(2,2) = 2;

  SMatrix<double,3> B = A; // save A in B
  cout << "A: " << endl << A << endl;

  A.sinvert();
  cout << "A^-1: " << endl << A << endl;

  // check if this is really the inverse:
  cout << "A^-1 * B: " << endl << A * B << endl;
\end{code}
The output is:
\begin{code}{blue}
A:
[            1           1           0
             1           2           0
             0           0           2 ]
A^-1:
[            2          -1           0
            -1           1           0
             0           0         0.5 ]
A^-1 * B:
[            1           0           0
             0           1           0
             0           0           1 ]
\end{code}
The \texttt{invert()} and \texttt{sinvert()} functions are derived from the
CERNLIB \texttt{dinv} and \texttt{dsinv} routines.
%
% Section
%
\section{Performance comparison}
\label{perf}
A performance comparison between the matrix and vector classes of
\texttt{Vt++} and \SMatrix\ has been carried out. The \texttt{Vt++}
implementation of matrix and vector classes is suffering from the C++
abstraction penalty (see section \ref{intro}), from dynamic storage allocation
and from run time dimension evaluation. So the
comparison shows nicely the impact of the template based implementation
of \SMatrix\ on the numerical performance.

\textbf{Note:} the performance depends on the dimension of vectors and
matrices. For small dimensions the impact of loop overheads and storage
allocation is dominant, whereas for large dimensions the vector or matrix
computation will dominate the processing time. As a Kalman filter based vertex
fit is based on low-dimensional vectors and matrices, a low dimension has been
chosen for the comparison.

Conditions:
\begin{enumerate}
  \item Intel Celeron 466 MHz, 256 MB RAM
  \item Compiler: g++ 2.95.2
  \item Compiler flags: g++ -O2 -funroll-loops -finline-functions
\end{enumerate}

\newcommand{\vt}{\textrm{\tiny Vt}}
\newcommand{\sm}{\textrm{\tiny SM}}
\begin{center}
  \setlength{\extrarowheight}{0.2ex}
  \setlength{\doublerulesep}{0pt}
  \begin{tabular}{l!{\vrule width 3pt\hspace{1ex}}r@{.}lr@{.}lr@{.}l}
    Operation ($\times 10^6$), $\textrm{dim} = 3$ & \multicolumn{2}{c}{$t_\vt$ [s]} &
    \multicolumn{2}{c}{$t_\sm$ [s]} &
    \multicolumn{2}{c}{$t_\vt/t_\sm$} \\\hline\hline\hline\hline
    dsinv() (CERNLIB, $A^{-1}$)    &   3&73    & 0&84  &  4&44\\
    dsfact() (CERNLIB, $\det A$)   &   2&98    & 0&53  &  5&62\\
    $C = A\cdot B$                 &   8&09    & 1&29  &  6&27\\
    $D = A + B\cdot C$             &  12&44    & 1&43  &  8&70\\
    $\vec{y} = A\cdot\vec{x}$      &   4&40    & 0&28  & 15&71\\
    $D = A + B + C$                &  11&29    & 0&43  & 26&25\\
    $D = -A + B + C$               &  13&27    & 0&47  & 28&23\\
    $a=\vec{x}^T\cdot A\cdot \vec{x}$ &   2&56    & 0&07  & 36&57\\
    $a=(\vec{x}+\vec{y})^T\cdot (A + B + C)\cdot (\vec{x}+\vec{y})$ &
    18&27 & 0&39 & 46&85\\\hline\hline
    -funroll-loops                 & \multicolumn{2}{c}{\tcr{$+3\%$}} &
    \multicolumn{2}{c}{\tcr{$+30\%$}} & \multicolumn{2}{c}{}
  \end{tabular}
\end{center}
\textbf{Note:} $t_\vt$ denotes the time consumption of the \texttt{Vt++} code
and $t_\sm$ the time consumption of the \SMatrix\ code. The third column shows the
ratio. As one can see, loop unrolling on the compiler level (initiated by the
\texttt{-funroll-loops} compiler flag) is much more efficient in case of the \SMatrix\
library than for \texttt{Vt++}. The reason for this is explained in section \ref{intro}.
%
% Section
%
\section{Vertexing}
The \SVertex\  class is a reimplementation of the Vt++ \texttt{Vertex}
class, based on the high performance vector and matrix classes \texttt{SVector}
and \SMatrix. The \texttt{SKalman} class represents a track in the
Kalman filter process. The \SVertex\  class implements a fixed sized
vertex object, that means the number of tracks in the vertex must be known
already at compile time. This is usually not a restriction for an analysis job,
as the signature of decaying particles is known in advance. The only limitation
comes from primary vertices, where the number of tracks depend on the data and
is not known at compile time.
%
% Subsection
%
\subsection{Track and Vertex abstraction}
\label{abstraction}
The \SVertex\  class is based on an abstraction of tracks and vertices
and thus allows to be interfaced to an arbitrary amount of (non curved) track
and vertex classes without changing a single line of \SVertex\ code.
The interface classes can be found in the following header files of the \BEE\
\texttt{interfaces} project:
\begin{description}
\item[TrackIf.hh] interface class for non curved tracks
\item[VertexIf.hh] interface class for vertices.
\end{description}
In other words: the \SVertex\  class can work with any track class which
is inherited from the \texttt{TrackIf} interface class:
\begin{code}{blue}
  #include "TrackIf.hh"

  class Line: public TrackIf {
    public:
    Line(double x, double y, ...);

    float x() const { return x_; };
    float y() const { return y_; };
    ...
    private:
    double x_, y_,...;
  }
\end{code}
The newly defined \texttt{Line} class has now the \texttt{TrackIf} interface
and can be used by \SVertex\ :
\begin{code}{blue}
  #include "Line.hh"
  #include "SVertex.hh"

  int main() {

    Line a(...);
    Line b(...);

    SVertex<2> vtx(a,b);
  }
\end{code}
The \SVertex\  class is inherited from the \texttt{VertexIf} interface,
thus allowing you to write code which is independent from the exact type of the
SVertex class and can be used for any class inherited from \texttt{VertexIf}:
\begin{code}{blue}
  class VertexIf;

  // function to return spatial distance between
  // VertexIf objects
  inline double Sdistance(const VertexIf& v1, const VertexIf& v2) {
    return mag(v1.vpos() - v2.vpos());
  }
\end{code}
Example:
\begin{code}{blue}
  int main() {
    
    SVertex<2> vtx1(...);
    SVertex<4> vtx2(...);
    
    vtx1.findVertexVt(); // fit vertex
    vtx2.findVertexVt(); // fit vertex

    // compute spatial distance
    double dist = Sdistance(vtx1,vtx2);
  }
\end{code}
\textbf{Note:} for a list of defined $\chi^2$-distance and spatial distance
functions between vertex, track and wire objects, see the reference manual or
the \texttt{\tcb{SDistance.hh}} header file.
%
% Subsection
%
\subsection{Vertex fit methods}
Four different methods to compute a vertex position are provided:
\begin{description}
  \item[\texttt{findVertexVt}] Kalman filter based vertex fit based on
    T. Lohses algorithm \cite{vt}. \textbf{Note:} before using this routine,
    make sure that the tracks are propagated to $z=0$.
  \item[\texttt{findVertex2D}] analytical vertex computation in 2
    dimensions. Returns the vertex position but doesn't change the internal
    state of the \SVertex\  object. The \texttt{calcVertex2D()} member
    function computes the vertex and sets the vertex positon.
  \item[\texttt{findVertex3D}] analytical vertex computation in 3 dimensions.
    Returns the result in a \texttt{SVector} object which must be given as an
    argument. It does not change the internal state of the \SVertex\ 
    object.
  \item[\texttt{EstimateVertex}] analytical vertex computation in 2 dimensions
    by using covariance matrix elements as weights. Returns result in a
    \texttt{SVector} object. The internal state of the \SVertex\  object
    is not changed.
\end{description}
In general, a vertex position can be set by hand using the \texttt{set\_vpos()}
member function:
\begin{code}{blue}
    SVertex<double,2> vtx(...);
    vtx.set_vpos(vtx.EstimateVertex());
\end{code}
\textbf{Note:} the covariance matrix of the vertex is only computed with the
Kalman filter based vertex fit.
%
% Subsection
%
\subsection{Computing the mother track}
The \SVertex\ class is inherited from the \texttt{TrackIf} class and is thus
able to represent the mother track\footnote{The incoming track is usually
  called \emph{mother track}.}, which can be computed from the vertex
position and the momentum sum of the outgoing tracks. For performance
reasons the mother track is never computed implicitly, but only if it is
explicitly wanted by the user. The \SVertex\ class is designed for high
performance and not for fool-proofness, therefore the user has to make sure
that the mother track has been calculated before accessing the
\texttt{TrackIf} member functions of an \SVertex\ object.

The following member functions exist to calculate the mother track:
\begin{description}
\item[\texttt{calc\_mother\_tr()}] calculate the track parameters ($t_x$, $t_y$,
  $p$) of the mother track by using the Kalman information of the tracks in the
  vertex. \textbf{Note:} the vertex must have been fitted with
  \texttt{findVertexVt()} when using this function.
\item[\texttt{calc\_mother\_cov()}] calculate the covariance matrix of the mother
  track. \textbf{Note:} the vertex must have been fitted with
  \texttt{findVertexVt()} when using this function.
\item[\texttt{calc\_mother()}] same as calling both member functions 
  \texttt{calc\_mother\_tr()} and \texttt{calc\_mother\_cov()}.
\item[\texttt{calc\_mother\_trtr()}] calculate the track parameters of the mother
  track by using the information from the track objects. No Kalman information
  is used, the vertex can be fitted with any routine (or set by hand).
\end{description}
A \texttt{SVector} with the track parameters $x$, $y$, $z$, $t_x$, $t_y$, $p$
is returned by the \texttt{mother()} member function.

Example:
\begin{code}{blue}
  SVertex<double,2> vtx1(...);

  // if fit is OK
  if(vtx1.findVertexVt() == true) {
    vtx1.calc_mother(); // compute mother track
    cout << "mother track: " << vtx1.mother() << endl;
  }
\end{code}
%
% Subsection
%
\subsection{Accessing track and Kalman information}
The \texttt{TrackIf} and \texttt{SKalman} information of tracks fitted to a vertex can be
accessed via the \texttt{track()} and \texttt{kalman()} information. The track
in question is specified by an index:
\begin{code}{blue}
  SVertex<double,4> vtx(...); // create vertex object

  // fit vertex
  if(vtx.findVertexVt() == true) {

    // get 3rd track
    const TrackIf* t3 = vtx.track(2);
    // get Kalman info of 3rd track
    const SKalman& k3 = vtx.kalman(2);

    // print Kalman information
    cout << "Kalman: " << k3 << endl;
  }
\end{code}
The \texttt{SKalman} class is inherited from \texttt{TrackIf}, too. Therefore
the Kalman filter objects represent the refitted tracks and can be used
in the same way as any other \texttt{TrackIf} based object.
\textbf{Note:} some of the \texttt{SKalman} functions are dummy functions in
order to comply with the \texttt{TrackIf} interface. See the reference manual
for more details.

Tracks can be set not only by calling the \SVertex\ constructor, but also
later:
\begin{code}{blue}
  SVertex<double,2> vtx(); // default CTOR

  TrackIf& t1 = ...; // get a track from somewhere
  TrackIf& t2 = ...; // get a track from somewhere

  vtx.track(0) = &t1; // set track pointer
  vtx.track(1) = &t2; // set track pointer
\end{code}
%
% Subsection
%
\subsection{Mass calculation}
Two member functions exist to compute an invariant mass of the vertex object
and one to compute an error of the invariant mass:
\begin{description}
\item[\texttt{mass}] computes the invariant mass using the Kalman filter
  information from the refitted track objects. \textbf{Note:} the vertex must
  have been fitted with \texttt{findVertexVt()} when using this function.
\item[\texttt{mass\_tr}] computes the invariant mass using the measured track momenta.
  There is no need to fit/compute the vertex when using this function.
\item[\texttt{massError}] computes an error of the invariant mass by a
  simplified formula which just takes the error of the refitted momentum into
  account. \textbf{Note:} the vertex must have been fitted with
  \texttt{findVertexVt()} when using this function.
\end{description}
In all cases, a \texttt{SVector} object which contains the (assumed) rest
masses of the daughter tracks must be given as an argument. The rest mass
vector must have the same dimension as the \SVertex\ object:
\begin{code}{blue}
  SVertex<double,3> vtx(...); // vertex with 3 tracks
  const double pimass = 0.134;
  const double Kmass  = 0.497;

  if(vtx.findVertexVt() == true) {
    // 1st track Kaon, others pion
    SVector<double,3> massvec(Kmass,pimass,pimass);

    // mass computed with refitted momenta
    double mass1 = vtx.mass(massvec);
    // mass computed with measured track momenta
    double mass2 = vtx.mass_tr(massvec);
    // mass error
    double masserr = vtx.massError(massvec);

    cout << "inv. mass: " << mass1 << ", " << mass2 
         << " err: " << masserr << endl;
  }
\end{code}
%
% Subsection
%
\subsection{Performance comparison to Vt++}
A performance comparison between the reimplemented vertex fit algorithms in
\SVertex\ and in Vt++ has been carried out. The implementations were compared
on a system with the same characteristics as in section \ref{perf}, the data used was
from the HERA-B year 2000 run, run 17137, reprocessing 2, Cloneremove
\cite{cloneremove} used.
\begin{center}
  \setlength{\extrarowheight}{0.4ex}
  \setlength{\doublerulesep}{0pt}
  \begin{tabular}{l!{\vrule width 3pt}rccc}
    fit algorithm  & events & $t_\vt$ & $t_\sm$ & $t_\vt/t_\sm$ \\\hline\hline\hline
    findVertex2D   & 15000 & 17m12.01s & 0m40.18s & \tcr{25.68}\\
    EstimateVertex & 15000 & 17m11.31s & 0m44.86s & \tcr{23.01}\\
    findVertex3D   & 15000 & 17m22.72s & 0m44.70s & \tcr{23.32}\\
    findVertexVt   &  1500 &  9m17.14s & 0m26.11s & \tcr{21.34}\\
    + mother track &  1500 & 10m14.87s & 0m37.04s & \tcr{16.60}
  \end{tabular}
\end{center}
\textbf{Note:} $t_\vt$ denotes the time consumption of the \texttt{Vt++} code
and $t_\sm$ the time consumption of the \SMatrix\ code. The third column shows the
ratio. In the last row, the Kalman filter based vertex fit including the
computation of the mother track (track parameters and covariance matrix) is compared.
%
% Section
%
\section{Extending the SMatrix package}
Most of the binary and unary operators in the \SMatrix\ library are generated
automatically by using the \texttt{\tcb{CreateUnaryOp.sh}} and the
\texttt{\tcb{CreateBinaryOp.sh}} shell scripts. Further operators can be
defined by adding definitions at the top of the scripts in the
\texttt{\tcb{OPLIST}} variable.

Example: In case we want to define a unary \texttt{cos()} operator it is
sufficient to add the line
\begin{code}{blue}
Cos,cos,cos
\end{code}
to the \texttt{OPLIST} variable in the \texttt{CreateUnaryOp.sh} script.
Executing the \texttt{CreateUnaryOp.sh} script regenerates the
\texttt{UnaryOperators.hh} header file with the new definitions. Now it is
possible to write code which takes the cosine element wise from a vector or
matrix:
\begin{code}{blue}
  SVector<double,3> x(0.1, 0.2, 0.3);
  cout << "cos: " << cos(x) << endl;
\end{code}
For binary operators, the analog steps have to be taken, now with the
\texttt{CreateBinaryOp.sh} script.
%
% Section
%
\section{Remarks}
\begin{enumerate}
\item At the moment, the \SVertex\ class does not provide mass constrained vertex
  fits, decay chain fits and pointing constrained vertex fits. The reason is that
  up to now the data quality didn't allow to make use of these features in an
  analysis. Due to manpower constraints a data-driven software development scheme
  evolved in the analysis framework \BEE.
\item The \SMatrix\ project is not available in the HERA-B reconstruction
  framework \texttt{ARTE} \cite{arte}. It is easily possible to make use of it in
  \texttt{ARTE} by writing an interface class which is inherited from
  \texttt{TrackIf} and interfaces RTRA to the \SVertex\ class. As already
  mentioned in sections \ref{intro} and \ref{abstraction}, such an interface can
  be established without changing a single line of code in the \SMatrix\ project.
\end{enumerate}
%
% Bibliography
%
\begin{thebibliography}{99}
\bibitem{Vt++} T. Glebe, Vt++ Version 1.0, HERA-B Note 00-175, Software 00-013
\bibitem{Grover} \texttt{http://wwwhera-b.mppmu.mpg.de/analysis/grover.html}
\bibitem{bee} \texttt{http://www.mpi-hd.mpg.de/herab/clue} or\\
  \texttt{/afs/desy.de/group/hera-b/BEE}
\bibitem{templ1} Todd L. Veldhuizen, C++ Templates as Partial Evaluation, 1999
  ACM SIGPLAN Workshop on Partial Evaluation and Semantics-Based Program
  Manipulation (PEPM'99).
\bibitem{alib} Todd L. Veldhuizen, Active Libraries: Rethinking the roles of
  compilers and libraries, SIAM Workshop on Object Oriented Methods for
  Inter-operable Scientific and Engineering Computing, October 21-23, 1998.
\bibitem{tmplmeta} Todd Veldhuizen and Kumaraswamy Ponnambalam, Linear algebra with C++ template metaprograms, Dr. Dobb's Journal, August 1996.
\bibitem{exprtmpl} Todd Veldhuizen, Expression Templates, C++ Report, June 1995.
\bibitem{exprtmpl2} Tood Veldhuizen and M. Ed Jernigan, Will C++ be faster than
  Fortran?, Proceedings of the 1st International Scientific Computing in Object Oriented Parallel Environments (ISCOPE'97)
\bibitem{vt} T. Lohse, Vertex Reconstruction And Fitting, HERA-B Note 95-013, Software 95-002.
\bibitem{cernlib} \texttt{http://wwwinfo.cern.ch/asd/cernlib}
\bibitem{arte} \texttt{http://www-hera-b.desy.de/subgroup/software}
\bibitem{cloneremove} M.-A. Pleier, Cloneremove V1.0, HERA-B Note 1-062,
  Software 01-009.
\end{thebibliography}
\end{document}

